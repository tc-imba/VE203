\documentclass{article}
\usepackage{enumerate}
\usepackage{amsmath}
\usepackage{amssymb}
\usepackage{graphicx}
\usepackage{subfigure}
\usepackage{geometry}
\usepackage{caption}
\geometry{left=3.0cm,right=3.0cm,top=3.0cm,bottom=4.0cm}
\renewcommand{\thesection}{Exercise 3.\arabic{section}}
\title{VE203 Assigment 3}
\author{Liu Yihao 515370910207}
\date{}
\begin{document}
\maketitle

\section{}
\begin{enumerate}[i)]
\item
\begin{enumerate}[(i)]
\item
$a\cdot(b\cdot c)=(a\cdot b)\cdot c$ for all $a,b,c\in S$
\item
$1\in S$ satisfies $a\cdot1=1\cdot a=a$ for all $a\in S$
\item
$|z|=1\Rightarrow z=x+yi(x^2+y^2=1)$
for every $a=x+yi\in S$ there exists an element $a^{-1}=x-yi\in S$ such that $a\cdot a^{-1}=a^{-1}\cdot a=1$
\end{enumerate}
So it is proved.

\item
\begin{enumerate}[(i)]
\item
$a\cdot(b\cdot c)=(a\cdot b)\cdot c$ for all $a,b,c\in S$
\item
$1\in S$ satisfies $a\cdot1=1\cdot a=a$ for all $a\in S$
\item
$z^n=1\Rightarrow|z|^n=1\Rightarrow|z|=1\Rightarrow z=x+yi(x^2+y^2=1)$\\
for every $a=x+yi\in S$ ($x^2+y^2=1$) there exists an element $a^{-1}=x-yi\in S$ such that $a\cdot a^{-1}=a^{-1}\cdot a=1$
\end{enumerate}
So it is proved.
\end{enumerate}

\section{}
\begin{enumerate}[i)]
\item
\begin{enumerate}[(i)]
\item
$A\cdot(B\cdot C)=(A\cdot B)\cdot C$ for all $A,B,C\in S$
\item
\begin{equation*}
E=\left(
\begin{array}{cc}
1 & 0\\
0 & 1\\
\end{array}
\right)
\end{equation*}
$E\in S$ satisfies $A\cdot E=E\cdot A=A$ for all $A\in S$
\item
\begin{equation*}
A^T=\left(
\begin{array}{cc}
\cos\varphi & \sin\varphi\\
-\sin\varphi& \cos\varphi\\
\end{array}
\right)\ \rm{When}\ \varphi=-\varphi_0
\end{equation*}
\begin{equation*}
A\cdot A^T=\left(
\begin{array}{cc}
\cos^2\varphi+\sin^2\varphi & 0\\
0 & \cos^2\varphi+\sin^2\varphi\\
\end{array}
\right)=\left(
\begin{array}{cc}
1 & 0\\
0 & 1\\
\end{array}
\right)
\end{equation*}
for every $A\in S$ there exists an element $A^{-1}=A^T\in S$ such that $A\cdot A^{-1}=A^{-1}\cdot A=E$
\end{enumerate}
So it is proved.

\item
\begin{enumerate}[(a)]
\item
\begin{enumerate}[(i)]
\item
$A\cdot(B\cdot C)=(A\cdot B)\cdot C$ for all $A,B,C\in SL$
\item
\begin{equation*}
E=\left(
\begin{array}{ccccc}
1 & 0 & \cdots & 0 & 0\\
0 & 1 & \cdots & 0 & 0\\
\cdots & \cdots & \cdots & \cdots & \cdots\\
0 & 0 & \cdots & 1 & 0\\
0 & 0 & \cdots & 0 & 1\\
\end{array}
\right)
\end{equation*}
$E\in SL$ satisfies $A\cdot E=E\cdot A=A$ for all $A\in SL$
\item
$$det(A)=det(A^T)=1$$
$$A\cdot A^T=E$$
for every $A\in SL$ there exists an element $A^{-1}=A^T\in SL$ such that $A\cdot A^{-1}=A^{-1}\cdot A=E$
\end{enumerate}
So it is proved.

\item
\begin{enumerate}[(i)]
\item
$A\cdot(B\cdot C)=(A\cdot B)\cdot C$ for all $A,B,C\in O$
\item
\begin{equation*}
E=\left(
\begin{array}{ccccc}
1 & 0 & \cdots & 0 & 0\\
0 & 1 & \cdots & 0 & 0\\
\cdots & \cdots & \cdots & \cdots & \cdots\\
0 & 0 & \cdots & 1 & 0\\
0 & 0 & \cdots & 0 & 1\\
\end{array}
\right)
\end{equation*}
$E\in O$ satisfies $A\cdot E=E\cdot A=A$ for all $A\in O$
\item
$$A\cdot A^T=E$$
for every $A\in O$ there exists an element $A^{-1}=A^T\in O$ such that $A\cdot A^{-1}=A^{-1}\cdot A=E$
\end{enumerate}
So it is proved.

\item
\begin{enumerate}[(i)]
\item
$A\cdot(B\cdot C)=(A\cdot B)\cdot C$ for all $A,B,C\in SO$
\item
\begin{equation*}
E=\left(
\begin{array}{ccccc}
1 & 0 & \cdots & 0 & 0\\
0 & 1 & \cdots & 0 & 0\\
\cdots & \cdots & \cdots & \cdots & \cdots\\
0 & 0 & \cdots & 1 & 0\\
0 & 0 & \cdots & 0 & 1\\
\end{array}
\right)
\end{equation*}
$E\in SO$ satisfies $A\cdot E=E\cdot A=A$ for all $A\in SO$
\item
$$det(A)=det(A^T)=1$$
$$A\cdot A^T=E$$
for every $A\in O$ there exists an element $A^{-1}=A^T\in SO$ such that $A\cdot A^{-1}=A^{-1}\cdot A=E$
\end{enumerate}
So it is proved.
\end{enumerate}

\end{enumerate}

\section{}
\begin{enumerate}[i)]
\item
reflexive: $2|a-a$ is true for all $a\in Z$\\
symmetric: if $2|a-b$ is true, then $2|b-a$ is true for all $a,b\in Z$\\
transitive: if $2|a-b$ and $2|b-c$ is true, then $2|a-b+b-c$ is true, so $2|a-c$ is true for all $a,b,c\in Z$
\item
$\lbrace2Z,2Z+1\rbrace$
\item
Let $m_1,m_2=m_1+2a\in m$, $n_1,n_2=n_1+2b\in n$, $2|n_1-n_2$, $2|m_1-m_2$\\
$$2|n_1-n_2+m_1-m_2\Longleftrightarrow2|(m_1+n_1)-(m_2+n_2)$$
$$[m]+[n]:=[m+n]$$
$$m_1n_1-m_2n_2=-4ab-2m_1b-2n_1a$$
$$2|-4ab-2m_1b-2n_1a$$
$$[m]\cdot[n]:=[m\cdot n]$$
\end{enumerate}

\section{}
Suppose $c=gcd(a,b)$, then $a=cm$, $b=cn$ where $m,n\in N^*$ and $gcd(m,n)=0$\\
$n=ax+by=(mx+ny)c$ where $mx+ny\in Z$\\
So all elements in T is integer multiples of $gcd(a,b)$\\
According to Theorem 1.6.7, $c|a$ and $c|b$ implies $c|(ax+by)$ for any $x,y\in Z$
So it is proved.

\section{}
When $n=3k$, $n^2=3(3k^2)$, which is divided\\
When $n=3k+1$, $n^2=3(3k^2+2k)+1$\\
When $n=3k+2$, $n^2=3(3k^2+4k+1)+1$\\
So it is proved.

\section{}
Suppose $c=gcd(a,a+n)$, then $a=qb$, $a+n=qc$\\
$n=q(c-b)$, so $c$ divides n\\
When $n=1$, $gcd(a,a+1)$ divides $1$, so $a$ and $a+1$ are always relatively prime.

\section{}
\begin{enumerate}[i)]
\item
\begin{align*}
72&=1\cdot56+16\\
56&=3\cdot16+8\\
16&=8\cdot2+0
\end{align*}
$$d=gcd(56,72)=8$$
$$8=56-3\cdot(72-56)=4\cdot56-3\cdot72$$
$$20\cdot56-15\cdot72=40$$
$$x=20+\frac{72}{8}t=20+9t$$
$$y=15-\frac{56}{8}t=20-7t$$
\item
\begin{align*}
439&=5\cdot84+19\\
84&=4\cdot19+8\\
19&=2\cdot8+3\\
8&=2\cdot3+2\\
3&=1\cdot2+1\\
\end{align*}
$$d=gcd(84,439)=1$$
$$1=3-2=3-(8-2\cdot3)=-8+3(19-2\cdot8)=3\cdot19-7(84-4\cdot19)
=-7\cdot84+31(439-5\cdot84)=31\cdot439-162\cdot84$$
$$-25272\cdot84-(-4836)\cdot439=156$$
$$x=-25272+439t$$
$$y=-4836+84t$$
\end{enumerate}

\section{}
\begin{enumerate}[i)]
\item
Since $gcd(a,b)=1|c$, we can apply Theorem 1.6.26\\
The general solution of $ax+by=c$ is
$$x=x_0+\frac{b}{d}$$
$$y=y_0-\frac{a}{d}$$
where $x_0,y_0$ is a solution to $ax+by=c$\\
Let $b'=-b$, the general solution of $ax-by=c$ is
$$x=-x_0-\frac{b}{d}$$
$$y=-y_0-\frac{a}{d}$$
\item
\begin{align*}
158&=2\cdot57+44\\
57&=1\cdot44+13\\
44&=3\cdot13+5\\
13&=2\cdot5+3\\
5&=1\cdot3+2\\
\end{align*}
$$d=gcd(158,57)=1$$
$$1=-5+2(13-2\cdot5)=2\cdot13-5(44-3\cdot13)=-5\cdot44+17(57-44)
=17\cdot57-22\cdot(158-2\cdot57)=-22\cdot158+61\cdot57$$
$$-154\cdot158-(-427)\cdot57=7$$
$$x=-154+57t$$
$$y=-4276+158t$$
\end{enumerate}

\section{}
\begin{enumerate}[i)]
\item
Suppose $a=3k_1+1$, $b=3k_2+1$, then
$$ab=(3k_1+1)(3k_2+1)=3(3k_1k_2+k_1+k_2)+1$$
Suppose a member of the set in not a prime, the number can be expressed by two members of the set. If either of the factor numbers isn't a prime, it can be expressed by another two members of the set. This procedure will last until all of the factor numbers are prime, so the number is a product of primes.
\item
$$100=10\cdot10=4\cdot25$$
\end{enumerate}

\section{}
\begin{enumerate}[i)]
\item
$(4k+3)|4\cdot(3\cdot7\cdots p)$, which means $4k+3|d+1$\\
Suppose there exist a prime of form $(4k+3)|d$, $gcd(d,d+1)\geqslant4k+3$, but according to Exercise 3.6, $gcd(d,d+1)=1$, so it is impossible, no prime of this form divides $d$
\item
$d=4\cdot(3\cdot7\cdots(p-1))+3$, which is in the form of $4k+3$\\
According to i), no prime of the form $4k+3$ divides $d$, so if it can be divided, the factors of $d$ can only be $4k+1$ (since it is an odd number). Suppose $a=4k_1+1,b=4k_2+1$, $ab=4(4k_1k_2+k_1+k_2)+1$, which is in the form of $4k+1$. So the product of numbers in the form of $4k+1$ will never be in the form of $4k+3$. It suggests that $d$, which is in the form of $4k+3$, can't be divided by $4k+1$
\item
According to i),ii), $d$ is an odd and no odd prime numbers divides $d$, which means $d$ is a prime number. Since $d>p$, we can choose $d$ as a prime number to form another $d'=4\cdot(3\cdot7\cdots d)$, and $d'$ is also a prime number. Repeat the procedure infinitely and we can get infinite number of primes of the form $4k+3$
\end{enumerate}
\end{document}
