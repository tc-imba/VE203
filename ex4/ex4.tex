\documentclass{article}
\usepackage{enumerate}
\usepackage{amsmath}
\usepackage{amssymb}
\usepackage{graphicx}
\usepackage{subfigure}
\usepackage{geometry}
\usepackage{caption}
\geometry{left=3.0cm,right=3.0cm,top=3.0cm,bottom=4.0cm}
\renewcommand{\thesection}{Exercise 4.\arabic{section}}
\title{VE203 Assigment 4}
\author{Liu Yihao 515370910207}
\date{}
\begin{document}
\maketitle

\section{}
\begin{enumerate}[i)]
\item
$$247=13\times19$$
$$3\times13-2\times19=1$$
\item
$$10^2\equiv3^2\ \rm{mod}\ 13$$
$$10^{100}\equiv3^{100}\ \rm{mod}\ 13$$
$$10^{100}\equiv3\cdot27^{33}\ \rm{mod}\ 13$$
$$27^{33}\equiv1^{33}\ \rm{mod}\ 13$$
$$10^{100}\equiv3\ \rm{mod}\ 13$$
\\
$$10^2\equiv9^2\ \rm{mod}\ 19$$
$$10^{100}\equiv9^{100}\ \rm{mod}\ 19$$
$$10^{100}\equiv9\cdot9^{3^{33}}\ \rm{mod}\ 19$$
$$10^{100}\equiv9\cdot7^{3^{11}}\ \rm{mod}\ 19$$
$$343^{11}\equiv1^{11}\ \rm{mod}\ 19$$
$$10^{100}\equiv9\ \rm{mod}\ 19$$
\item
$$19\times11\ \rm{mod}\ 13=1$$
$$13\times3\ \rm{mod}\ 19=1$$
$$3\times19\times11+9\times13\times3=978$$
$$978\equiv237\ \rm{mod}\ 247$$
\end{enumerate}

\section{}
$$4^n\equiv7\ \rm{mod}\ 9$$
$$4^n\equiv9\ \rm{mod}\ 11$$
$$9\times5\ \rm{mod}\ 11=1$$
$$11\times5\ \rm{mod}\ 9=1$$
$$7\times11\times5+9\times9\times5=790$$
$$790\equiv4^n\ \rm{mod}\ 99$$
$$4^n=790+99k\ (k\in Z,k>-8)$$
$$n=8\ \rm{is\ a\ solution\ to\ the\ equation.}$$

\section{}
$$45029^2<2027651281<45030^2$$
$$\sqrt{(45030+11)^2-2027651281}=1020$$
$$2027651281=(45041-1020)(45041+1020)=44021\times46061$$

\section{}
$$5^6\equiv1\ \rm{mod}\ 7$$
$$5^{2003}\equiv5^{6^{333}}\times5^5\ \rm{mod}\ 7$$
$$5^{2003}\equiv3\ \rm{mod}\ 7$$
$$5^{10}\equiv1\ \rm{mod}\ 11$$
$$5^{2003}\equiv5^{10^{200}}\times5^3\ \rm{mod}\ 11$$
$$5^{2003}\equiv4\ \rm{mod}\ 7$$
$$5^{12}\equiv1\ \rm{mod}\ 13$$
$$5^{2003}\equiv5^{12^{166}}\times5^{11}\ \rm{mod}\ 13$$
$$5^{2003}\equiv8\ \rm{mod}\ 13$$
$$11\times13\times5\ \rm{mod}\ 7=1$$
$$7\times13\times4\ \rm{mod}\ 11=1$$
$$7\times11\times12\ \rm{mod}\ 13=1$$
$$3\times11\times13\times5+4\times7\times13\times4+8\times7\times11\times12=10993$$
$$10993\equiv983\ \rm{mod}\ 1001$$

\section{}
\begin{enumerate}[i)]
\item
$$(p-1)!\equiv -1\ \rm{mod} \ p$$
$$(p-1)!\equiv p-1\ \rm{mod} \ p$$
$$(p-2)!\equiv 1\ \rm{mod} \ p$$
If $p$ is not a prime and $p>3$, then there must exist $k\in[2,p-2],k\in N$ and $k$ mod $p=0$,\\
so $(p-2)!\equiv 0\ \rm{mod} \ p$, which reaches a contradiction.\\
If $p=2$ or $p=3$, it is obvious that $(p-1)!\equiv -1\ \rm{mod} \ p$.
\item
$$2z=m-1$$
$$z+1=m-z\equiv-z\ \rm{mod}\ m$$
$$z+k=m-z-k+1\equiv-z-k+1\ \rm{mod}\ m\ ,\ k\in[1,z]$$
$$(z+1)(z+2)\cdots2z\equiv(-1)^zz!\ \rm{mod}\ m$$
$$z!(z+1)(z+2)\cdots2z\equiv(-1)^z(z!)^2\ \rm{mod}\ m$$
$$(m-1)!\equiv(-1)^z(z!)^2\ \rm{mod}\ m$$
\item
When $p=4k+1,k\in N$,
$$(p-1)!\equiv(-1)^{2k}(2k!)^2\ \rm{mod}\ p$$
p is a prime when
$$(2k!)^2\equiv-1\ \rm{mod}\ p$$
When $p=4k+3,k\in N$,
$$(p-1)!\equiv(-1)^{2k+1}(2k+1!)^2\ \rm{mod}\ p$$
p is a prime when
$$(2k+1!)^2\equiv1\ \rm{mod}\ p$$

\end{enumerate}

\section{}
\begin{enumerate}[i)]
\item
$$1^2\equiv1\ \rm{mod}\ 11$$
$$2^2\equiv4\ \rm{mod}\ 11$$
$$3^2\equiv9\ \rm{mod}\ 11$$
$$4^2\equiv5\ \rm{mod}\ 11$$
$$5^2\equiv3\ \rm{mod}\ 11$$
$$6^2\equiv3\ \rm{mod}\ 11$$
$$7^2\equiv5\ \rm{mod}\ 11$$
$$8^2\equiv9\ \rm{mod}\ 11$$
$$9^2\equiv4\ \rm{mod}\ 11$$
$$10^2\equiv1\ \rm{mod}\ 11$$
So $1+11k,3+11k,4+11k,5+11k,9+11k,k\in N$ are quadratic residues of 11.
\item
Suppose $p=2k+1,k\in N$, $x=p-b,b\in[1,2k],b\in N$
$$(p-b)^2=(p)^2-2pb+b^2$$
$$(p-b)^2\equiv b^2\ \rm{mod}\ p$$
$$b^2\equiv b^2\ \rm{mod}\ p$$
Since p is an odd number, $p-b\neq b$,\\
so $x=b$ and $x=p-b$ are two incongruent solutions if $b^2\equiv a\ \rm{mod}\ p$,\\
or there is no solution if $b^2\not\equiv a\ \rm{mod}\ p$,
\item
According to ii), we can find that when $x=b$ or $x=p-b$, the value of $a$ is the same.\\
Let $b\in[1,k]$, then $b<p-b$, and let $n\in[1,k-b],n\in N$,\\
then for $b\in[1,k-1]$, if for $x=b$ and $x=b+n$, suppose the value of $a$ is the same,
$$b^2\equiv(b+n)^2\ \rm{mod}\ p$$
$$(2b+n)n\equiv0\ \rm{mod}\ p$$
$$2b+n\leqslant2b+k-b=k+b<p$$
$$n<p$$
Since $p$ is a prime number, $(2b+n)n\not\equiv0\ \rm{mod}\ p$, which reaches a contradiction.\\
So for any two $b$ the value of $a$ isn't the same, there are exactly $\frac{p-1}{2}$ quadratic residues of p among the integers 1,2,...,$p-1$.
\item
Let $c\in[1,k]$, then 
$$x^2\equiv a\equiv b\equiv c^2\ \rm{mod}\ p$$
If $a$ is a quadratic residue of $p$, we can find $c$ so that $x=c$ and $x=p-c$ are two incongruent solutions, and $b$ is also a quadratic residue of $p$, then
$$\left(\frac{a}{p}\right)=\left(\frac{b}{p}\right)=1$$
If $a$ isn't a quadratic residue of $p$, we can't find $c$ so that $x=c$ and $x=p-c$ are two incongruent solutions, and $b$ is also not a quadratic residue of $p$, then
$$\left(\frac{a}{p}\right)=\left(\frac{b}{p}\right)=-1$$
\item
If $a$ is a quadratic residue of $p$, $(\frac{a}{p})=1$, then let $a=x^2+kp,k\in Z$
$$a^{\frac{p-1}{2}}=(x^2+kp)^{\frac{p-1}{2}}=\sum_{i=0}^{\frac{p-1}{2}}(x^2)^i+(kp)^{\frac{p-1}{2}-i}\equiv(x^2)^{\frac{p-1}{2}}\ \rm{mod}\ p$$
$$a^{\frac{p-1}{2}}\equiv x^{p-1}\ \rm{mod}\ p$$
$$a^{\frac{p-1}{2}}\equiv 1\ \rm{mod}\ p$$
$$\left(\frac{a}{p}\right)\equiv a^{\frac{p-1}{2}}\ \rm{mod}\ p$$
If $a$ isn't a quadratic residue of $p$, $(\frac{a}{p})=-1$
$$a^{p-1}\equiv1\ \rm{mod}\ p$$
$$a^{\frac{p-1}{2}}\equiv\pm1\ \rm{mod}\ p$$
According to the above, we can easily get that if $a^{\frac{p-1}{2}}\equiv1\ \rm{mod}\ p$, $a$ is a quadratic residue of $p$, so $a^{\frac{p-1}{2}}\equiv-1\ \rm{mod}\ p$ here.
$$\left(\frac{a}{p}\right)\equiv a^{\frac{p-1}{2}}\ \rm{mod}\ p$$
\item
$$\left(\frac{ab}{p}\right)\equiv (ab)^{\frac{p-1}{2}}\equiv a^{\frac{p-1}{2}}\cdot b^{\frac{p-1}{2}}\equiv\left(\frac{a}{p}\right)\left(\frac{b}{p}\right)\ \rm{mod}\ p$$
Since $p$ is an odd prime $(p\geqslant3)$\\
When $\left(\frac{ab}{p}\right)=1$, $\left(\frac{a}{p}\right)\left(\frac{b}{p}\right)=1+kp$, so 
$$\left(\frac{ab}{p}\right)=\left(\frac{a}{p}\right)\left(\frac{b}{p}\right)=1$$
When $\left(\frac{ab}{p}\right)=-1$, $\left(\frac{a}{p}\right)\left(\frac{b}{p}\right)=-1+kp$, so 
$$\left(\frac{ab}{p}\right)=\left(\frac{a}{p}\right)\left(\frac{b}{p}\right)=-1$$
\item
If $a$ is a negative integer in v), we can simply get the same conclusion
$$\left(\frac{a}{p}\right)\equiv a^{\frac{p-1}{2}}\ \rm{mod}\ p$$
When $p=4k+1,k\in N$,
$$(-1)^{\frac{4k+1-1}{2}}=(-1)^{2k}=1$$
$$\left(\frac{a}{p}\right)\equiv1\ \rm{mod}\ p$$
Using the method in vi), we can find $\left(\frac{a}{p}\right)=1$, so -1 is a quadratic residue of $p$.\\
When $p=4k+3,k\in N$,
$$(-1)^{\frac{4k+3-1}{2}}=(-1)^{2k+1}=-1$$
$$\left(\frac{a}{p}\right)\equiv-1\ \rm{mod}\ p$$
Using the method in vi), we can find $\left(\frac{a}{p}\right)=-1$, so -1 isn't a quadratic residue of $p$.
\item
Let $x^2=35k+29,k\in N$
$$x^2\equiv4\ \rm{mod}\ 5$$
$$x^2\equiv1\ \rm{mod}\ 7$$
\begin{equation*}
\left\lbrace
\begin{array}{cc}
x\equiv&2\ \rm{mod}\ 5\\
x\equiv&1\ \rm{mod}\ 7\\
\end{array}
\right.\ or\ 
\left\lbrace
\begin{array}{cc}
x\equiv&2\ \rm{mod}\ 5\\
x\equiv&-1\ \rm{mod}\ 7\\
\end{array}
\right.\ or\ 
\left\lbrace
\begin{array}{cc}
x\equiv&-2\ \rm{mod}\ 5\\
x\equiv&1\ \rm{mod}\ 7\\
\end{array}
\right.\ or\ 
\left\lbrace
\begin{array}{cc}
x\equiv&-2\ \rm{mod}\ 5\\
x\equiv&-1\ \rm{mod}\ 7\\
\end{array}
\right.
\end{equation*}
$$7\times3\ \rm{mod}\ 5=1$$
$$5\times3\ \rm{mod}\ 7=1$$
$$x_1=[(2\times7\times3+1\times5\times3)\ \rm{mod}\ 35]+35k=22+35k$$
$$x_2=[(2\times7\times3-1\times5\times3)\ \rm{mod}\ 35]+35k=27+35k$$
$$x_3=[(-2\times7\times3+1\times5\times3)\ \rm{mod}\ 35]+35k=8+35k$$
$$x_4=[(-2\times7\times3-1\times5\times3)\ \rm{mod}\ 35]+35k=13+35k$$
\end{enumerate}

\end{document}

